O AE começou a ser construído a partir de seus \textbf{genes}, os quais compõem populações e determinam suas ações. Eles possuem comportamentos programados, então eles devem ser capazes de reagir a diferentes processos evolutivos por conta própria. Da perspectiva do problema, assim como na natureza, não é possível ver como um gene se comporta, apenas como ele se expressa num membro da população.

Genes devem permitir todas as ações possíveis dentro de seu espectro de ações iniciais. Por exemplo, se o gene possuir uma expressividade inteira de 0 a 5, todos estes números devem ser alcançáveis ao longo de gerações suficientes.