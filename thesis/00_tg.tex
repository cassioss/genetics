% !TEX encoding = UTF-8 Unicode
% -*- coding: UTF-8; -*-
% vim: set fenc=utf-8
%\documentclass[12pt]{book}
\documentclass[12pt,plainheader,pnumplain]{abnt}
\usepackage[brazil,brazilian]{babel}
\usepackage[utf8x]{inputenc}
\usepackage[T1]{fontenc}
\usepackage{textcomp}
\usepackage{mathtools}
\usepackage{amssymb}
\usepackage[portuguese,ruled]{algorithm2e}
\usepackage{url}
\usepackage[num]{abntcite}
\usepackage{acronym}
\usepackage{hyperref}
\usepackage{listings}
\usepackage{framed}
\usepackage{xcolor}
\usepackage{pgfplots}
\usepackage{pdfpages}
\usepackage{bookmark}
\usepackage{enumitem}
\usepackage{cite}
\usepackage{multirow,booktabs}

% início da configuração de listagens
\lstset{basicstyle=\small, frame=single, breaklines=true, showstringspaces=false}

\lstset{emph={%  
    MOVE24, DELAY, GOTO%
    },emphstyle={\color{blue}\bfseries}%
}

\renewcommand{\lstlistingname}{Listagem}
\renewcommand{\lstlistlistingname}{Lista de Listagens}
\newcommand{\oes}{\~oes }
\newcommand{\cao}{\c c\~ao }
\newcommand{\coes}{\c c\~oes }
\newcommand{\src}{src/experiments}

% fim da configuração de listagens

% A setting that would be applied to all pgfplots
\pgfplotsset{every axis/.append style={
        scaled y ticks = false, 
        scaled x ticks = false, 
        y tick label style={/pgf/number format/.cd, fixed, fixed zerofill,
                            int detect,1000 sep={\;},precision=3},
        x tick label style={/pgf/number format/.cd, fixed, fixed zerofill,
                            int detect, 1000 sep={},precision=3}
    }
}

\begin{document}

\includepdf{capa_tg.pdf}
\bookmark[level=0, page=1]{Capa}

\includepdf{folha_rosto_tg.pdf}
\bookmark[level=0, page=2]{Folha de Rosto}

\includepdf{verso_folha_rosto_tg.pdf}
\bookmark[level=0, page=3]{Verso da Folha de Rosto}

\includepdf{folha_aprovacao_tg.pdf}
\bookmark[level=0, page=4]{Folha de Aprova\cao}

\vspace*{18cm}

\begin{flushright}
\begin{minipage}[t]{7.0 cm}
Dedico este trabalho à minha mãe Lucinez, que me trouxe suporte durante todo meu tempo no ITA, e ao Instituto Ismart, oportunidade que muda a minha vida desde 2006.
\end{minipage}
\end{flushright}
\thispagestyle{empty}
\bookmark[level=0, page=5]{Dedicat\'oria}

\chapter*{Agradecimentos}
� minha fam�lia, que se esfor�ou tanto para que eu ficasse at� o fim no ITA.

Ao meu professor orientador Carlos Forster, que me guiou por caminhos tortuosos at� a conclus�o deste trabalho de gradua��o.

Aos meus colegas de turma, que foram uma fam�lia a mais enquanto estive no ITA, e que despertaram em mim o desejo de sempre voar mais alto.

Aos demais professores da COMP, que nos acompanham desde antes do Curso Profissional, e que podem nos ver hoje como verdadeiros engenheiros.

E ao Instituto Ismart, que investe em minha forma��o acad�mica, profissional e pessoal desde 2006 com muito amor e carinho.

\newpage
\thispagestyle{empty}

\vspace*{15.0 cm}

\textit{When you are inspired by some great purpose, some extraordinary project, all your thoughts break their bounds. Your mind transcends limitations, your consciousness expands in every direction and you find yourself in a new, great and wonderful world. Dormant forces, faculties and talents become alive, and you discover yourself to be a greater person by far than you ever dreamed yourself to be.}

\begin{flushright}

Pata�jali, criador das pr�ticas do Ioga.

\end{flushright}
\thispagestyle{empty}
\bookmark[level=0, page=7]{Cita\cao}

\chapter*{Resumo}
Este trabalho focou no uso de Algoritmos Evolutivos (AE), mais especificamente de um Algoritmo Genético (AG), para a resolução de diferentes problemas. Tais algoritmos trabalham com populações onde a menor parte de um indivíduo dela é o gene, o qual contém informação suficiente para que o indivíduo tente resolver o problema. O AG desenvolvido foi utilizado para três problemas: OneMax Booleano, com genes expressados por 0 ou 1; OneMax Real, com genes expressados no intervalo [0, 1); e uma variação do Problema do Caixeiro Viajante que permite utilizar atalhos entre as cidades. De modo a otimizar o AG, utilizou-se duas implementações extras. A primeira delas foi a de elitismo entre as gerações, mantendo o melhor indivíduo imune a variações. A segunda foi o uso não só da versão estática do AG, na qual os parâmetros de entrada são fixos durante a execução, mas também de um módulo extra chamado Algoritmo Genético Adaptativo (AGA), o qual foi adicionado ao código do AG e permite que parâmetros de variação (recombinação e mutação) mudem durante a execução. A conclusão deste trabalho foi a de que, por se utilizar a mesma implementação do AG para diferentes problemas (com exceção das entradas específicas de cada problema), o tratamento que facilitaria a convergência de um problema para a(s) melhore(s) resposta(s) acabou não sendo o melhor para outros problemas. Em termos de otimização, o uso do AGA se mostrou eficaz no encontro de boas soluções, e deve ser dado mais atenção a ele. Para trabalhos futuros, sugere-se, quando o foco for trabalhar em um problema específico, moldar o AG para se beneficiar desta escolha ao máximo, e utilizar também alguma versão de AGA em sua implementação.

\thispagestyle{empty}

\chapter*{Abstract}
This work focused on using Evolutionary Algorithms (EA), more specifically Genetic Algorithms (GA), to solve different problems. Such algorithms work with populations where the smallest portion of an individual is the gene, which contains enough information for the individual to try to solve the problem. The GA developed was used for three problems: Boolean OneMax, with genes expressed by 0 or 1; Real OneMax, with genes expressed in the interval [0, 1); and a variation of the Traveling Salesman Problem, in which it's possible to go through shortcuts between the cities. In order to optimize the GA, two extra implementations were used. The first one was the elitism between generations, keeping the best individual immune to variations. The second one was using not only a static version of the GA, in which the input parameters are fixed during the execution, but also an extra module called Adaptive Genetic Algorithm (AGA), which was added to the GA's code and allows variation parameters (recombination and mutation) to change during the execution. The conclusion of this work was that, since the same GA was used to different problems (with the exception of problem-specific inputs), the treatment that would ease convergence for one problem towards the best solution(s) is not the same for other problems. In terms of optimizations, using an AGA turned out to be effective for finding good solutions. For future works, we suggest, when the focus is to solve a specific problem, to mold the GA to benefit from this choice to the fullest, and use some version of AGA in its implementation, too.

\thispagestyle{empty}

\tableofcontents
\bookmark[level=0, page=10]{Sum\'ario}
\addtocontents{sumario}{\protect\thispagestyle{empty}}
\thispagestyle{empty}
% TODO: Descobrir como tirar numeração do Sumário

\listoffigures
\thispagestyle{empty}
% TODO: Descobrir como tirar numeração da lista de figuras

\listoftables
\thispagestyle{empty}

\listofalgorithms
\thispagestyle{empty}

\lstlistoflistings
\thispagestyle{empty}

\chapter*{Lista de Abreviaturas}
\begin{acronym}
	\acro{AE}{Algoritmo Evolutivo}
	\acro{AG}{Algoritmo Gen\'etico}
	\acro{AGA}{Algoritmo Gen\'etico Adaptativo}
\end{acronym}
\thispagestyle{empty}

\chapter{Introdu\cao}
\label{1_introducao}

A resolução de diversos problemas se dá na forma de algoritmos, de instruções bem definidas. No entanto, alguns algoritmos podem pedir inúmeras instruções até concluírem, o que pode até mesmo inviabilizar a solução encontrada. A Inteligência Artificial pode atuar sobre tais problemas de modo a interagir com o problema e aprender com ele a encontrar uma solução. Ótimos candidatos para esta tarefa são os chamados \emph{Algoritmos Evolutivos (AE)}.

Algoritmos evolutivos são aqueles que se baseiam nos princípios de evolução natural da Biologia, e são aplicados em um modo particular de solução de problemas: o da tentativa-e-erro \cite{eiben2003introduction}. Tais algoritmos seguem um framework mais ou menos comum, atuante sobre diferente \emph{gerações} de um problema, por meio de mudanças e combinações de \emph{indivíduos} existentes numa \emph{população}. Tal framework, conjuntamente com as ações evolutivas, pode ser visto na figura \ref{fig:evolution-framework}.

\begin{figure}[ht!]
    \centering \includegraphics[width=1.0\textwidth]{evolution-framework.png}
    \caption{Framework de um algoritmo evolutivo.}
    \label{fig:evolution-framework}
\end{figure}

Cada indivíduo está tentando resolver o problema durante a execução do algoritmo evolutivo. A população contém estes indivíduos e é o alvo de interesse do processo evolutivo, e uma geração é a população que sobrevive após um ciclo de processos evolutivos do \ac{AE} sobre o problema. Similar ao processo evolutivo, seus componentes principais são as operações de variação (mutação e recombinação) e de seleção (seleção da geração pai e sobrevivência), chamadas aqui simplesmente de \emph{operações de evolução} \cite{eiben2011parameter}.

Este trabalho se utilizou de um grupo específico de Algoritmo Evolutivo, chamado de \textbf{Algoritmo Genético (AG)}. Um AG possui, como a menor estrutura de seus indivíduos, o \emph{gene}. Um gene costuma ter apenas duas propriedades: uma expressividade, normalmente associada a um valor numérico, e uma forma de mudá-la aleatoriamente.

Para um AG, a \emph{mutação} é uma mudança não controlada de um indivíduo feita a partir da mudança na expressividade de um ou mais genes. A \emph{recombinação} envolve a mistura de genes vindos de dois indivíduos que são cruzados. A \emph{seleção} envolve uma escolha a dedo dos melhores exemplares para cruzamento (a geração pai). A \emph{sobrevivência} envolve a rejeição de indivíduos que não estejam aptos o suficiente para resolver o problema. Tal aptidão é normalmente associada a uma função definida conjuntamente com o problema, chamada de \emph{função de fitness}.

\section{Objetivo}

Este trabalho propôs implementar um algoritmo genético capaz de resolver problemas determinados de diferentes complexidades e encontrar boas soluções após uma quantidade razoável de gerações. Apesar de se considerar performance, a prioridade aqui foi buscar boas soluções.

\section{Abordagem}

Em termos de implementação, o AG deve ser tal que, uma vez aplicado sobre um problema e uma população, as gerações se desenvolvam automaticamente. O trabalho foi então dividido em 4 etapas:

\begin{enumerate}[label={[\arabic*]}]
	\item Escolha e implementação de problemas compatíveis com a aplicação do AE;
	\item Implementação do AG e de suas operações de evolução;
	\item Otimização do AG;
	\item Análise de performance e coleta de dados.
\end{enumerate}

As primeiras duas etapas são interdependentes, e precisaram ser completadas primeiro e conjuntamente. As demais etapas foram feitas sequencialmente, e foi encima da última etapa que as conclusões foram feitas.

De modo a permitir a reutilização deste código em outros problemas, o AG foi feito de modo a ser compatível com mais de um problema. As ferramentas de análise e coleta de dados consideraram tanto métricas comuns da literatura quanto parâmetros específicos dos problemas abordados.

Optou-se por utilizar Python como linguagem principal do código feito para este trabalho.

\section{Plano de Trabalho}
\label{sec:plano_trabalho}

Este trabalho foi planejado de forma que as etapas de otimização e análise do AE demorassem mais tempo. A validação dos resultados foi feita com base na literatura encontrada e nos resultados obtidos por algoritmos de código aberto (\emph{open source}) aplicados aos mesmos problemas.

\section{Organização do Trabalho}

\begin{itemize}
	\item \textbf{Capítulo 1:} Introdução
	\item \textbf{Capítulo 2:} Problemas Escolhidos
	\item \textbf{Capítulo 3:} Algoritmo Genético
	\item \textbf{Capítulo 4:} Otimização do Algoritmo
	\item \textbf{Capítulo 5:} Análises e Resultados
	\item \textbf{Capítulo 6:} Conclusões e Trabalhos Futuros
\end{itemize}


\chapter{Problemas Escolhidos}
\label{2_problemas}

Por mais que a estrutura básica de um algoritmo evolutivo seja capaz de resolver múltiplos problemas, é importante que ele seja validado por problemas de diferentes naturezas. Para isso, este trabalho focou sua atenção na resolução de três problemas com implementações diferentes para a construção do algoritmo genético.

Como o AG atua diretamente com populações, um problema deve defini-las de antemão, tanto em termos de indivíduo quanto em termos de gene. Conjuntamente, é necessário definir quão apto um indivíduo está frente à solução que ele propõe. Isso é feito por meio da \emph{função de fitness}. Para o número de indivíduos na população (ao menos inicialmente), utilizou-se um valor padrão de 100 indivíduos.

Três problemas foram escolhidos: OneMax Booleano \cite{giguere1998population}, OneMax Real e uma variação do problema do Caixeiro Viajante \cite{applegate2011traveling}. Os dois primeiros foram escolhidos em termos não só de sua facilidade de implementação, mas também porque são problemas "fáceis" em termos de encontro de solução ótima (como será detalhado a seguir), o que permite testar hipóteses e heurísticas de modo bem mais simples. O terceiro problema foi escolhido por não só ter uma literatura rica, mas também por ser um problema complexo, cujos resultados podem ser analisados e comparados de modo mais rico.

\section{OneMax Booleano}

Dado um conjunto de 100 bits iniciados em 0, o AG deve ser capaz de tornar todos os bits iguais a 1.

\subsection*{Gene}

Utilizou-se aqui o \emph{BooleanGene}, um gene com expressividade booleana (0 ou 1). Sua operação de mutação consiste numa operação semelhante a jogar cara-e-coroa, trocando o valor expresso para 0 ou 1 aleatoriamente, com igual probabilidade.

\subsection*{Indivíduo}

Cada indivíduo tentará resolver o problema, o que faz com que cada indivíduo tenha 100 genes do tipo BooleanGene.

\subsection*{Função de fitness}

Conta-se o número de genes de um indivíduo que sejam iguais a 1. Quanto maior a contagem, melhor.

\section{OneMax Real}

Dado um conjunto de 100 variáveis reais iniciadas em 0, o AG deve ser capaz de tornar todas elas o mais próximo de 1. Este OneMax possui uma caminhada bem mais lenta que o anterior, pois um gene booleano possui apenas dois estados, o que faz com que as ações de mutação permitam uma evolução muito mais rápida, enquanto um gene real muda sua expressividade num espectro bem maior.

\subsection*{Gene}

Utilizou-se aqui o \emph{RealGene}, um gene com expressividade real entre 0.0 e 1.0. Sua operação de mutação consiste numa escolha aleatória de um número real no intervalo [0.0, 1).

\subsection*{Indivíduo}

Cada indivíduo tentará resolver o problema, o que faz com que cada indivíduo tenha 100 genes do tipo RealGene.

\subsection*{Função de fitness}

Soma-se a expressividade de todos os genes de um indivíduo. Quanto maior a contagem, melhor. Feita de maneira apropriada, esta função pode ser a mesma utilizada para o OneMax Booleano.

\section{Caixeiro Viajante (Variação)}

Dado um conjunto de cidades e as distâncias entre elas, o AG deve ser capaz de descobrir qual o menor caminho que possibilita a um caixeiro visitar todas as cidades e retornar à cidade original. Tal problema é NP-Hard, e avaliar se uma solução candidata é algo tão complexo quanto a resolução do problema em si.

Este problema é uma variação do original por não obrigar ao problema que as cidades sejam visitadas uma única vez. Isso permite que um indivíduo sugira um atalho entre duas cidades.

\subsection*{Gene}

Utilizou-se o \emph{IntegerGene}, um gene com expressividade inteira entre 0 e K-1, com uma operação de mutação capaz de escolher aleatoriamente um valor inteiro neste intervalo. A inicialização deste gene possui K como parâmetro.

\subsection*{Indivíduo}

No caso de um indivíduo do problema do Caixeiro Viajante, foi pensado que o mesmo deveria ser capaz de gerar, a partir da expressividade de seus genes, um percurso que passasse uma única vez por todas as cidades. Para isso, os genes aqui foram organizados de modo um pouco diferente dos problemas OneMax.

Digamos, por exemplo, que um caixeiro na cidade A precise passar pelas cidades [B, C, D, E, F] e voltar à cidade A. O indivíduo de tal problema teria então quatro genes (o número total de cidades, menos dois) criados da seguinte forma:

\begin{itemize}
	\item O primeiro gene possui expressividade de 0 a 4;
	\item O segundo gene possui expressividade de 0 a 3;
	\item O terceiro gene possui expressividade de 0 a 2;
	\item O quarto gene possui expressividade de 0 a 1.
\end{itemize}

Digamos que um dos indivíduos do AG tenha, pela expressividade de seus genes, os valores [3, 0, 1, 0]. Para se calcular o percurso feito por tal indivíduo, escolhe-se a cidade da lista naquela posição, a qual é removida antes de se escolher a próxima cidade. Ou seja:

\begin{itemize}
	\item Gene 1: [3] mapeia a cidade E na lista [B, C, D, E, F]. Sem ela, a lista se torna [B, C, D, F];
	\item Gene 2: [0] mapeia a cidade B na lista [B, C, D, F]. Sem ela, a lista se torna [C, D, F];
	\item Gene 3: [1] mapeia a cidade D na lista [C, D, F]. Sem ela, a lista se torna [C, F];
	\item Gene 4: [0] mapeia a cidade C na lista [C, F]. Sem ela, a lista se torna [F].
\end{itemize}

Como [F] foi a única cidade que tais genes não escolheram, ela será visitada por último. Com isso, o indivíduo com genes [3, 0, 1, 0] traz o percurso $A \rightarrow E \rightarrow B \rightarrow D \rightarrow C \rightarrow F \rightarrow A$.

O percurso inicial terá sempre genes com expressividade zero. No exemplo fornecido, o caminho inicial (trazido por [0, 0, 0, 0]) de todos os indivíduos seria $A \rightarrow B \rightarrow C \rightarrow D \rightarrow E \rightarrow F \rightarrow A$.

\subsection*{Função de fitness}

A função de fitness aqui calcula a distância percorrida pelo caixeiro no trajeto completo trazido pelo indivíduo, considerando sempre o menor caminho a ser percorrido entre quaisquer duas cidades. Isso é trazido pelo uso do algoritmo de Dijkstra \cite{dijkstra1959note} no grafo constituído pelas cidades. Seu uso será detalhado melhor a seguir, mas quanto menor a distância que um indivíduo encontrar, melhor.

Como parte do problema do Caixeiro Viajante é o de encontrar um trajeto de menor custo, não é dito à função de fitness qual é a menor distância que o grafo possui.

\subsection*{Algoritmo de Dijkstra}

O algoritmo de Dijkstra possui o seguinte pseudocódigo \cite{cormen2001dijkstra}:

\begin{algorithm}[H]
$\textbf{Dijkstra(} Grafo, cidade \textbf{)}$
\Begin{
	$Inicializar(Q)$\;
	$Inicializar(dist)$\;
	$Inicializar(prev)$\;
	\ForEach{cidade v no Grafo} {
		$dist[v] \gets \infty$\;
		$prev[v] \gets desconhecido$\;
		$Q.adicionar(v)$\;
	}
	$dist[cidade] \gets 0$\;
	\While{Q não estiver vazio} {
		$u \gets VerticeEmQComMenorDist(Q, dist)$\;
		$Q.remover(u)$\;
		\ForEach{Vizinho w de u} {
			$atalho \gets dist[u] + DistanciaEntre(u, w)$\;
			\If{atalho < dist[w]} {
				$dist[w] \gets atalho$\;
				$prev[w] \gets u$\;
			}
		}
	}
	\Return{dist, prev}
}
\caption{Pseudocódigo do Algoritmo de Dijkstra.}
\label{alg:dijkstra}
\end{algorithm}

A complexidade de tal algoritmo, por possuir um loop dentro de outro, é, para o pior caso, $O(N^2)$, sendo $N$ o número de cidades. No entanto, ele só descobre a menor distância tendo como referência a cidade utilizada como parâmetro. Por conta disso, o algoritmo precisa ser rodado uma vez para cada cidade, trazendo uma complexidade total $O(N^3)$ para o seu uso.

O requerimento para este algoritmo convergir é o de que a distância entre quaisquer duas cidades seja maior que zero. Para ser utilizada com o AG, foi considerado também que o grafo fosse conexo, de tal forma que qualquer percurso sugerido por um indivíduo tenha uma distância total finita e não-nula.

O algoritmo de Dijkstra tenta buscar atalhos entre quaisquer duas cidades em sua execução. Se ele for utilizado para completar o grafo, um percurso como $A \rightarrow B$ pode pedir que sejam visitadas outras cidades (o que, em casos reais de mapeamento de cidades, é comum acontecer).

Um indivíduo que propuser um percurso não saberá dizer se há atalhos entre duas cidades. No entanto, como estamos procurando a menor distância a ser percorrida no grafo, a função de fitness irá visitar todos os atalhos, sem considerar tal cidade como visitada. Esperou-se que, ao longo das gerações, os atalhos acabassem sendo escolhidos para visita naturalmente.


\chapter{Algoritmo Gen\'etico}
\label{3_algoritmo_genetico}

O pseudocódigo tradicional de um algoritmo evolutivo pode ser dado por \cite{algoritmopseudo}:

\begin{algorithm}[H]
\Begin{
$P \gets InicializarPopulacao(fitness)$\;
$P.fitness()$\;
$t = 0$\;
\While{t < número de gerações} {
	$Pais \gets Selecao(P)$\;
	$Filhos \gets Crossover(Pais)$\;
	$P \gets P \cup Filhos$\;
	$Mutacao(P)$\;
	$P.fitness()$\;
	$P \gets Sobrevivem(P)$\;
}
}
\caption{Pseudocódigo do Algoritmo Genético.}
\label{alg:ag}
\end{algorithm}

Cada uma destas funções principais (todas exceto a função de fitness, que é uma das entradas do problemas) pode ser feita de diferentes maneiras. No entanto, tais implementações independem do problema a ser resolvido, o que torna sua confecção e manutenção bem mais simples.

\section{Seleção}

A escolha dos pais é feita por.

\section{Crossover}

Escolhidos os pais, optou-se neste trabalho por embaralhá-los aleatoriamente numa lista e, escolhendo-os dois a dois, verificar se os dois pais serão cruzados, de acordo com uma probabilidade $p_c$. Se sim, os pais terão seus materiais genéticos misturados por uma ação chamada \emph{crossover}.

Crossover envolve a quebra da sequência de genes de dois indivíduos em um mesma secção, com a subsequente troca de material na região delimitada por esta secção. Este trabalho se utilizou do crossover em dois pontos, o qual divide as sequências em duas partes distintas, ocorrendo troca do material genético entre estas partes. Tal ação é mostrada na figura \ref{fig:evolution-framework}.

\begin{figure}[ht!]
    \centering \includegraphics[width=1.0\textwidth]{crossover.png}
    \caption{Ação de crossover em dois pontos.}
    \label{fig:crossover}
\end{figure}

Toda ação de crossover gera duas sequências de genes (filhas) novas a partir das sequências originais (pais), adicionando dois indivíduos novos a cada crossover bem sucedido.

\section{Mutação}

A ação de mutação funciona, em sua estrutura mais básica, da seguinte forma: itera-se sobre todos os genes da população um a um. Com uma probabilidade $p_m$, é avaliado se o gene deveria ou não alterar sua expressividade. Se sim, o gene tem seu valor alterado aleatoriamente.

\section{Sobrevivência}

Texto.



\chapter{Otimização do Algoritmo}
\label{4_otimizacao}

Como proposta deste trabalho, pensou-se em como seria possível otimizar um AG de modo a encontrar boas soluções em poucas gerações. Dado o pseudocódigo de um AE, é possível propor uma série de otimizações, desde aquelas voltadas à melhoria de processamento, como o processamento paralelo de indivíduos em uma dada geração, àquelas que otimizam cada uma das quatro operações evolutivas principais (seleção, recombinação, mutação e sobrevivência).

De modo geral, o que traz novas soluções ao problema são as operações de variação (recombinação e mutação), e por conta disso, foram as mais analisadas neste capítulo. Otimizações mais simples discutidas em outros trabalhos, mas que se mostraram eficazes no encontro ou manutenção de boas soluções, também foram discutidas neste capítulo.

A manutenção de boas soluções foi obtida pelo uso de \emph{elitismo}. A otimização nas operações de variação foi obtida pelo uso dos chamados Algoritmos Genéticos Adaptativos (AGA).

\section{Elitismo}

O elitismo é a manutenção do indivíduo mais adaptado de uma geração, deixando-o imune a mutações para que a melhor solução não seja perdida \cite{mitchell1998introduction}. Um indivíduo elitista ainda pode ser considerado para recombinação e geração de filhos, uma vez que as operações de mutação e recombinação são independentes. É possível também criar um grupo elitista, mantendo-se uma certa quantidade ou porcentagem de indivíduos imune a mutações.

Este trabalho utilizou elitismo para o melhor indivíduo em todas as execuções do AG. Tal propriedade pode ser desativada no código.

\section{Algoritmo Genético Adaptativo (AGA)}

A forma mais tradicional de implementação de um AG atribui valores estáticos aos parâmetros de entrada, incluindo os parâmetros de crossover e mutação. No entanto, os indivíduos buscarão soluções de acordo com estes dois parâmetros, e deixá-los estáticos pode limitar o alcance do AG e impedi-lo de encontrar soluções melhores.

Se fosse possível modificar tais parâmetros enquanto o AG é executado, de modo a se adaptar às mudanças de fitness dos próprios indivíduos, teríamos uma solução. Um bom candidato para isso são os chamados Algoritmos Genéticos Adaptivos (AGAs) \cite{srinivas1994adaptive}.

O conceito por trás de um AGA envolve implementar em cima de um AG de modo a modificar os parâmetros de crossover e/ou mutação ao longo do tempo. Não obstante, é possível moldar um AGA de modo a tratar crossover e mutação com probabilidades diferentes para cada indivíduo, de acordo com seus valores de fitness.

Para este trabalho, optou-se por trabalhar com versões adaptadas de outros AGAs \cite{jakobovic1999adaptive, wang2001improved, srinivas1994adaptive} e implementar uma versão própria, explicada a seguir:

\begin{itemize}

	\item Apenas o parâmetro de mutação é modificado ao longo das gerações, uma vez que o crossover seja sempre um valor alto (como 0.9, padronizado neste trabalho);

	\item A adaptação de $p_m$ acontece apenas depois que um ciclo de operações de evolução acontecer;

	\item O que decidirá se $p_m$ mudará será o desvio do melhor valor de fitness $f_{best}$ em comparação com o fitness médio $\bar{f}$, como na equação a seguir:

\begin{equation}
	\left| \frac{f_{best} - \bar{f}}{\bar{f}} \right|
\label{eq:aga}
\end{equation}

	\item Se este desvio for menor que um valor $p{p_m}_0$, isso significa que a mutação está fraca e os indivíduos estão se aproximando de uma mesma solução, que pode não ser necessariamente a melhor. Para contornar isso, $p_m$ irá aumentar;

	\item Caso contrário, as soluções estarão se desviando muito, o que pode ser resultado de uma mutação intensa. Para resolver isso, $p_m$ irá diminuir;

	\item O valor de ${p_m}_0$ é o valor inicial de $p_m$ para o valor inicial de ${p_m}_0$, de modo a servir de termômetro para o valor do desvio;

	\item No entanto, $p_m$ não pode ser igual a zero nem maior que 1, dado que representa uma probabilidade. Seguindo a linha de outros trabalhos \cite{matthias2013variable}, $p_m$ será limitado ao intervalo [0.001, 0.5] (se $p_m$ tentar extrapolar estes limites, ele retornará ao valor extremo mais próximo);

	\item O incremento/decremento para $p_m$ será linear e igual a 0.001;

	\item Como há uma divisão por $\bar{f}$, se este valor for zero ou muito próximo de zero para alguma geração, este AGA não será executado.

\end{itemize}

Traduzindo-se a explicação para um algoritmo, chegamos ao código mostrado no algoritmo \ref{alg:aga}. Este trabalho avaliará o desempenho deste AGA comparando a evolução da população com e sem o uso do AGA para um mesmo valor inicial de $p_m$. O intuito não foi o de encontrar um AGA ideal, mas sim o de avaliar se o uso dele ajudaria ou não no encontro de soluções melhores.

\begin{algorithm}[ht]
\Begin{
	${p_m}_0 \gets $ (valor inicial de $p_m$ na inicialização do AG)\;
	$\epsilon \gets 0.0001$\;
	\ForEach{ciclo de operações de evolução} {
		$\bar{f} \gets $ (média dos valores de fitness)\;
		\If{$\bar{f} < \epsilon$} {
			\Return
		}
		$f_{best} \gets $ (melhor valor de fitness na população)\;
		$desvio \gets \left| \frac{f_{best} - \bar{f}}{\bar{f}} \right|$\;
		\If{desvio <= ${p_m}_0$} {
			$p_m \gets min(0.5, p_m + 0.001)$\;
		}
		\Else{
			$p_m \gets max(0.001, p_m - 0.001)$\;
		}
	}
}
\caption{Pseudocódigo do Algoritmo Genético Adaptativo (AGA).}
\label{alg:aga}
\end{algorithm}

Para exemplificar o funcionamento do AGA, foram feitas simulações para diferentes valores de ${p_m}_0$ para o OneMax Booleano, na figura \ref{fig:aga_test}. É possível ver que $p_m$ tenta sempre manter o desvio entorno de ${p_m}_0$, incentivando a busca de soluções diferentes para o sistema.

\begin{figure}[ht!]
    \centering \includegraphics[width=1.0\textwidth]{boolean_onemax_aga.jpg}
    \caption{Evolução dos valores de $p_m$ e ${p_m}_0$ para o problema OneMax para valores diferentes de ${p_m}_0$ (em vermelho) ao longo de 500 gerações.}
    \label{fig:aga_test}
\end{figure}

A ideia por trás de uma implementação própria foi a de testar a implementação de um AGA a partir de conceitos mais simples. Se a ideia de adaptação de um AGA, conforme vista na literatura, for tão simples quanto a base evolutiva do AG, o formato dele também deverá buscar uma implementação simples.



\chapter{An\'alises e Resultados}
\label{5_resultados}

\section{Parâmetros analisados}

Foram escolhidos parâmetros simples para analisar a evolução feita ao longo das gerações:

\begin{itemize}
	\item Valor mínimo de fitness em um indivíduo;
	\item Valor máximo de fitness em um indivíduo;
	\item Valor médio de fitness entre todos os indivíduos;
	\item Desvio padrão dos valores de fitness.
\end{itemize}

O desvio padrão aqui foi calculado por:

\begin{equation}
	\sigma = \sqrt{\frac{1}{N-1} \sum_{i=1}^N (x_i - \overline{x})^2}
\end{equation}

\section{OneMax Booleano}

\begin{figure}[ht!]
    \centering \includegraphics[width=1.0\textwidth]{onemax_boolean.jpg}
    \caption{Evolução do fitness para o problema do OneMax Booleano com mínimo, máximo e valor médio, com $p_c=0.9$ e $p_m=0.01$.}
    \label{fig:onemax_boolean}
\end{figure}

\begin{figure}[ht!]
    \centering \includegraphics[width=1.0\textwidth]{onemax_boolean_std.jpg}
    \caption{Desvio padrão ao longo das gerações para o problema do OneMax Booleano, com $p_c=0.9$ e $p_m=0.01$.}
    \label{fig:onemax_boolean}
\end{figure}

\section{OneMax Real}

\begin{figure}[ht!]
    \centering \includegraphics[width=1.0\textwidth]{onemax_real.jpg}
    \caption{Evolução do fitness para o problema do OneMax Real com mínimo, máximo e valor médio, com $p_c=0.9$ e $p_m=0.01$.}
    \label{fig:onemax_boolean}
\end{figure}

\begin{figure}[ht!]
    \centering \includegraphics[width=1.0\textwidth]{onemax_real_std.jpg}
    \caption{Desvio padrão ao longo das gerações para o problema do OneMax Real, com $p_c=0.9$ e $p_m=0.01$.}
    \label{fig:onemax_boolean}
\end{figure}

Texto aqui.

\chapter{Conclus\oes e Trabalhos Futuros}
\label{6_conclusoes}

\section{Conclusões}

Este trabalho teve como objetivo o desenvolvimento de um Algoritmo Genético (AG) para a resolução de diferentes problemas de modo otimizado. Foram escolhidos três problemas: OneMax Booleano (variável booleana), OneMax Real (variável real) e Caixeiro Viajante Adaptado (permitindo atalhos). De modo a melhorar a performance do AG, foram implementadas duas otimizações: o elitismo do melhor indivíduo, e o uso de um Algoritmo Genético Adaptativo (AGA) com implementação própria, focada na adaptação do parâmetro de mutação.

Chegou-se à conclusão que utilizar um mesmo algoritmo para resolver um grupo diferente de problemas sem perda de performance é virtualmente impossível. No entanto, o uso do AGA implementado se mostrou mais eficiente no encontro de melhores soluções para o problema do Caixeiro Viajante, um problema NP-Hard. Por conta disso, o uso deste AGA é fortemente incentivado em outros problemas, buscando sempre validações e melhorias. Se isso não for possível, o conceito por trás do desenvolvimento de um algoritmo adaptativo é muito forte e próximo do conceito básico do AG, e seu uso deve ser considerado em outros algoritmos evolutivos.

\section{Trabalhos Futuros}

O desenvolvimento feito neste trabalho resultaram na criação de um algoritmo adaptativo e de implementações próprias tanto do algoritmo genético quanto dos problemas escolhidos. A proposta deste trabalho foi audaz, e muitos conceitos foram desenvolvidos ao mesmo tempo, conceitos que precisam ser quebrados e maturados ainda mais. Sugere-se então as seguintes frentes de trabalho futuro:

\begin{itemize}

	\item Análise mais cuidadosa do problema do OneMax Real, considerando modelagens matemáticas e sua proximidade com o OneMax Booleano;

	\item Análise do problema do Caixeiro Viajante com a distribuição genética deste trabalho, comparada com implementações mais tradicionais (este trabalho distribuiu os genes deste problema de tal forma que o crossover de dois pontos ainda fosse possível);

	\item Evolução dos conceitos por trás do AGA desenvolvido aqui, buscando formalizar os conceitos por trás de seu desenvolvimento e, onde for possível, melhorá-lo (tanto para performance quanto para busca de soluções);

	\item Testar o AGA junto a outros problemas, para verificar prós, contras e limitações;

	\item O algoritmo em si foi pouco trabalhado (mesmo tendo muita coisa sendo discutida aqui), dado o tempo utilizado para este trabalho. Recomenda-se uma melhoria drástica deste algoritmo de modo a comportar outras implementações dos problemas, dos operadores de evolução e de outros algoritmos adaptativos.

\end{itemize}


\bibliography{referencias}
\bibliographystyle{plain}

\appendix

\newpage
\chapter{Genes Utilizados}
\label{appendix:genes}
\lstinputlisting[language={python}, caption={Genes Utilizados (genes.py).}]{\src/genes.py}

\newpage
\chapter{Indivíduos dos problemas}
\label{appendix:genes}
\lstinputlisting[language={python}, caption={Indivíduos dos problemas (individuals.py).}]{\src/individuals.py}

\newpage
\chapter{Algoritmo Genético}
\label{appendix:geneflow}
\lstinputlisting[language={python}, caption={Algoritmo Genético (geneflow.py).}]{\src/geneflow.py}

\newpage
\chapter{Funções de fitness}
\label{appendix:fitness}
\lstinputlisting[language={python}, caption={Funções de fitness (fitness.py).}]{\src/fitness.py}

\newpage
\chapter{Mapa das cidades - Caixeiro Viajante}
\label{appendix:dijks}
\lstinputlisting[language={python}, caption={Arquivo com mapa das cidades - Caixeiro Viajante (dijkstra17.py).}]{\src/dijkstra17.py}

\newpage
\chapter{Implementação do Algoritmo de Dijkstra}
\label{appendix:dijks}
\lstinputlisting[language={python}, caption={Arquivo com algoritmo de Dijkstra (dijkstra.py).}]{\src/dijkstra.py}

\newpage
\chapter{Funções de utilidade}
\label{appendix:fitness}
\lstinputlisting[language={python}, caption={Funções de utilidade (utils.py)}]{\src/utils.py}

\includepdf{folha_registro_documento.pdf}
\bookmark[level=0, page=75]{Folha de Registro do Documento}

\end{document}