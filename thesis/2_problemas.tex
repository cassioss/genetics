\label{2_problemas}

Por mais que a estrutura básica de um algoritmo evolutivo seja capaz de resolver múltiplos problemas, é importante que ele seja validado por problemas de diferentes naturezas. Para isso, este trabalho focou sua atenção na resolução de três problemas com implementações diferentes para a construção do algoritmo genético.

Como o AG atua diretamente com populações, um problema deve defini-las de antemão, tanto em termos de indivíduo quanto em termos de gene. Conjuntamente, é necessário definir quão apto um indivíduo está frente à solução que ele propõe. Isso é feito por meio da \emph{função de fitness}. Para o número de indivíduos na população (ao menos inicialmente), utilizou-se um valor padrão de 100 indivíduos.

Três problemas foram escolhidos: OneMax Booleano \cite{giguere1998population}, OneMax Real e uma adaptação do problema do Caixeiro Viajante \cite{applegate2011traveling}. Os dois primeiros foram escolhidos em termos não só de sua facilidade de implementação, mas também porque são problemas "fáceis" em termos de encontro de solução ótima (como será detalhado a seguir), o que permite testar hipóteses e heurísticas de modo bem mais simples. O terceiro problema foi escolhido por não só ter uma literatura rica, mas também por ser um problema complexo, cujos resultados podem ser analisados e comparados de modo mais rico.

\section{OneMax Booleano}

Dado um conjunto de 100 bits iniciados em 0, o AG deve ser capaz de tornar todos os bits iguais a 1.

\subsection*{Gene}

Utilizou-se aqui o \emph{BooleanGene}, um gene com expressividade booleana (0 ou 1). Sua operação de mutação consiste numa operação semelhante a jogar cara-e-coroa, trocando o valor expresso para 0 ou 1 aleatoriamente, com igual probabilidade.

\subsection*{Indivíduo}

Cada indivíduo tentará resolver o problema, o que faz com que cada indivíduo tenha 100 genes do tipo BooleanGene.

\subsection*{Função de fitness}

Conta-se o número de genes de um indivíduo que sejam iguais a 1. Quanto maior a contagem, melhor.

\section{OneMax Real}

Dado um conjunto de 100 variáveis reais iniciadas em 0, o AG deve ser capaz de tornar todas elas o mais próximo de 1. Este OneMax possui uma caminhada bem mais lenta que o anterior, pois um gene booleano possui apenas dois estados, o que faz com que as ações de mutação permitam uma evolução muito mais rápida, enquanto um gene real muda sua expressividade num espectro bem maior.

\subsection*{Gene}

Utilizou-se aqui o \emph{RealGene}, um gene com expressividade real entre 0.0 e 1.0. Sua operação de mutação consiste numa escolha aleatória de um número real no intervalo [0.0, 1).

\subsection*{Indivíduo}

Cada indivíduo tentará resolver o problema, o que faz com que cada indivíduo tenha 100 genes do tipo RealGene.

\subsection*{Função de fitness}

Soma-se a expressividade de todos os genes de um indivíduo. Quanto maior a contagem, melhor. Feita de maneira apropriada, esta função pode ser a mesma utilizada para o OneMax Booleano.

\section{Caixeiro Viajante Adaptado}

Dado um conjunto de cidades e as distâncias entre elas, o AG deve ser capaz de descobrir qual o menor caminho que possibilita a um caixeiro visitar todas as cidades e retornar à cidade original. Tal problema é NP-Hard, e avaliar se uma solução candidata é algo tão complexo quanto a resolução do problema em si.

Este problema é uma adaptação do original por não obrigar ao problema que as cidades sejam visitadas uma única vez. Isso permite que um indivíduo possa atravessar duas cidades através de um atalho que passe por outras cidades (com isso, o grafo não precisa ser completo).

\subsection*{Gene}

Utilizou-se o \emph{IntegerGene}, um gene com expressividade inteira entre 0 e K-1, com uma operação de mutação capaz de escolher aleatoriamente um valor inteiro neste intervalo. A inicialização deste gene possui K como parâmetro.

\subsection*{Indivíduo}

No caso de um indivíduo do problema do Caixeiro Viajante, foi pensado que o mesmo deveria ser capaz de gerar, a partir da expressividade de seus genes, um percurso que passasse uma única vez por todas as cidades. Para isso, os genes aqui foram organizados de modo um pouco diferente dos problemas OneMax.

Digamos, por exemplo, que um caixeiro na cidade A precise passar pelas cidades [B, C, D, E, F] e voltar à cidade A. O indivíduo de tal problema teria então quatro genes (o número total de cidades, menos dois) criados da seguinte forma:

\begin{itemize}
	\item O primeiro gene possui expressividade de 0 a 4;
	\item O segundo gene possui expressividade de 0 a 3;
	\item O terceiro gene possui expressividade de 0 a 2;
	\item O quarto gene possui expressividade de 0 a 1.
\end{itemize}

Digamos que um dos indivíduos do AG tenha, pela expressividade de seus genes, os valores [3, 0, 1, 0]. Para se calcular o percurso feito por tal indivíduo, escolhe-se a cidade da lista naquela posição, a qual é removida antes de se escolher a próxima cidade. Ou seja:

\begin{itemize}
	\item Gene 1: [3] mapeia a cidade E na lista [B, C, D, E, F]. Sem ela, a lista se torna [B, C, D, F];
	\item Gene 2: [0] mapeia a cidade B na lista [B, C, D, F]. Sem ela, a lista se torna [C, D, F];
	\item Gene 3: [1] mapeia a cidade D na lista [C, D, F]. Sem ela, a lista se torna [C, F];
	\item Gene 4: [0] mapeia a cidade C na lista [C, F]. Sem ela, a lista se torna [F].
\end{itemize}

Como [F] foi a única cidade que tais genes não escolheram, ela será visitada por último. Com isso, o indivíduo com genes [3, 0, 1, 0] traz o percurso $A \rightarrow E \rightarrow B \rightarrow D \rightarrow C \rightarrow F \rightarrow A$.

O percurso inicial terá sempre genes com expressividade zero. No exemplo fornecido, o caminho inicial (trazido por [0, 0, 0, 0]) de todos os indivíduos seria $A \rightarrow B \rightarrow C \rightarrow D \rightarrow E \rightarrow F \rightarrow A$.

\subsection*{Função de fitness}

A função de fitness aqui calcula a distância percorrida pelo caixeiro no trajeto completo trazido pelo indivíduo, considerando sempre o menor caminho a ser percorrido entre quaisquer duas cidades. Isso é trazido pelo uso do algoritmo de Dijkstra \cite{dijkstra1959note} no grafo constituído pelas cidades. Seu uso será detalhado melhor a seguir, mas quanto menor a distância que um indivíduo encontrar, melhor.

Como parte do problema do Caixeiro Viajante é o de encontrar um trajeto de menor custo, não é dito à função de fitness qual é a menor distância que o grafo possui.

\subsection*{Algoritmo de Dijkstra}

O algoritmo de Dijkstra possui o seguinte pseudocódigo \cite{cormen2001dijkstra}:

\begin{algorithm}[H]
$\textbf{Dijkstra(} Grafo, cidade \textbf{)}$
\Begin{
	$Inicializar(Q)$\;
	$Inicializar(dist)$\;
	$Inicializar(prev)$\;
	\ForEach{cidade v no Grafo} {
		$dist[v] \gets \infty$\;
		$prev[v] \gets desconhecido$\;
		$Q.adicionar(v)$\;
	}
	$dist[cidade] \gets 0$\;
	\While{Q não estiver vazio} {
		$u \gets VerticeEmQComMenorDist(Q, dist)$\;
		$Q.remover(u)$\;
		\ForEach{vizinho w de u} {
			$atalho \gets dist[u] + DistanciaEntre(u, w)$\;
			\If{atalho < dist[w]} {
				$dist[w] \gets atalho$\;
				$prev[w] \gets u$\;
			}
		}
	}
	\Return{dist, prev}
}
\caption{Pseudocódigo do Algoritmo de Dijkstra.}
\label{alg:dijkstra}
\end{algorithm}

A complexidade de tal algoritmo, por possuir um loop dentro de outro, é, para o pior caso, $O(N^2)$, sendo $N$ o número de cidades. No entanto, ele só descobre a menor distância tendo como referência a cidade utilizada como parâmetro. Por conta disso, o algoritmo precisa ser rodado uma vez para cada cidade, trazendo uma complexidade total $O(N^3)$ para o seu uso.

O requerimento para este algoritmo convergir é o de que a distância entre quaisquer duas cidades seja maior que zero, e que a distância de uma cidade para ela mesma (se existir) seja zero. Para ser utilizada com o AG, considerou-se também um grafo inicial conexo, de tal forma que qualquer percurso sugerido por um indivíduo fosse sempre possível.

O algoritmo de Dijkstra buscará atalhos entre as cidades. Um indivíduo que propuser um trajeto, a princípio, não saberá dizer se há atalhos. Se houver um, ele será apresentado na tela, mas a função de fitness não irá considerar as cidades do atalho como visitadas. Esperou-se que, ao longo das gerações, tais cidades fossem escolhidas naturalmente pela população.
