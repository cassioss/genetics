\label{3_algoritmo_genetico}

O pseudocódigo tradicional de um algoritmo evolutivo pode ser dado por \cite{algoritmopseudo}:

\begin{algorithm}[H]
$\textbf{AG(} fitness \textbf{)}$
\Begin{
	$P \gets InicializarPopulacao(fitness)$\;
	$P.fitness()$\;
	$t = 0$\;
	\While{t < número de gerações} {
		$Pais \gets Selecao(P)$\;
		$Filhos \gets Crossover(Pais)$\;
		$P \gets P \cup Filhos$\;
		$Mutacao(P)$\;
		$P.fitness()$\;
		$P \gets Sobrevivem(P)$\;
	}
}
\caption{Pseudocódigo de um Algoritmo Evolutivo.}
\label{alg:ae}
\end{algorithm}

Cada uma destas funções principais pode ser feita de diferentes maneiras. No entanto, tais implementações independem do problema a ser resolvido, o que torna sua confecção e manutenção bem mais simples.

Para que este trabalho não fugisse muito de seu escopo, que é o de resolver problemas, as decisões feitas para o AG tentaram ser as mais simples possíveis frente à literatura existente.

\section{Parâmetros de entrada}

O AG possui uma série de algoritmos de entrada. No entanto, excetuando-se os parâmetros específicos de cada problema (como os associados à função de fitness ou ao tipo de indivíduo que comporá a população), restam um total de 4 parâmetros:

\begin{itemize}
	\item O número de indivíduos na população ($\mu$);
	\item O número de gerações ($NGEN$);
	\item A probabilidade de crossover ($p_c$);
	\item A probabilidade de mutação ($p_m$).
\end{itemize}

As variáveis $p_c$ e $p_m$ serão discutidas a seguir. O valor de $NGEN$ variou de acordo com o problema e com sua complexidade, mas foi padronizado um valor de 200 gerações para cada execução do AG. O tamanho da população $\mu$ ficou padronizado como 100.

\section{Seleção dos pais}

A escolha dos pais foi feita ordenando a geração em função de seu valor de fitness, com os pais sendo pareados sequencialmente, dois a dois.

\section{Recombinação / Crossover}

Escolhidos os pais, avalia-se, com probabilidade $p_c$, se eles serão cruzados. Se sim, seus materiais genéticos serão misturados por uma ação chamada \emph{crossover}.

O crossover envolve a troca de genes entre dois indivíduos. Este trabalho se utilizou do crossover de dois pontos, o qual divide as sequências em duas partes distintas, ocorrendo troca do material genético entre estas partes. Tal ação é mostrada na figura \ref{fig:crossover}.

\begin{figure}[ht!]
    \centering \includegraphics[width=1.0\textwidth]{crossover.png}
    \caption{Ação de crossover em dois pontos.}
    \label{fig:crossover}
\end{figure}

Toda ação bem sucedida de crossover gera duas sequências de genes (filhas) a partir das sequências originais (pais), as quais são transformadas em indivíduos e adicionadas à população.

O valor normalmente associado à variável $p_c$ na literatura costuma ficar entre 0.6 e 0.95 \cite{dejong1975analysis, eiben2011parameter, obitkopc}. Padronizou-se o valor de 0.9 neste trabalho. Ele precisa ser alto para que os filhos tenham maiores chances de variar seus genes e sugerir melhores respostas ao longo das gerações.

Os problemas deste trabalho foram organizados de tal forma que os genes de cada indivíduo fossem capazes de interagir com o problema separadamente. Por conta disso, o crossover de dois pontos pode ser aplicado sem problemas nos indivíduos.

\section{Mutação}

A ação de mutação atua sobre todos os genes da população (pais e filhos), a qual avalia, com probabilidade $p_m$, se a expressividade destes genes será alterada. Se sim, o valor deste gene é alterado de acordo com a assinatura de mutação do gene, tal como foi explicado no capítulo anterior. Normalmente, esta mudança de valor é aleatória.

O valor associado à variável $p_m$ deve ser baixo, pois um valor muito alto não seria capaz de permitir a convergência das soluções \cite{dejong1975analysis, eiben2011parameter, obitkopc}. O valor padronizado neste trabalho foi de 0.01.

\section{Sobrevivência}

A ação de sobrevivência aqui também tentou ser relativamente simples. Os indivíduos (pais e filhos) após a mutação são ordenados de acordo com a função de fitness, e apenas os $\mu$ melhores indivíduos são mantidos. Ou seja, a população não aumenta de tamanho ao longo das gerações.

