In Robotics, there is a special interest in humanoid robots, because these are theoretically the most suitable to reuse infrastructure made for humans. However, controlling a humanoid robot with high degree of freedom is a complex task. Many methods were proposed to solve the problem of humanoid locomotion; in general, these methods can be classified in 2 major approaches: ``model-based'' and ``model-free''. The main objective of this work is to generate a gait for a Hitec Robonova-I robot using optimization techniques. Therefere, we use a model-free approach, where fairly simple parametrized models, based on Truncated Fourier Series (TFS), are applied to generate joints angular trajectories. To find a parameters set that generates a fast and stable walk, optimization algorithms are used, in specific Genetic Algorithm (GA) and Particle Swarm Optimization (PSO). Since training directly on the real robot would be extremely inconvenient and could lead to hardware damage, the process was done in simulation first and the walk learnt was adapted to the real robot later. To allow use of simulation, a simulated model of Robonova-I was made inside the USARSim simulator. So, tests were made to evaluate the resulting walks and it was verified that the best walk from this work is faster than than the ones publicly available for Robonova-I. Later, to provide an additional validation, the same process was made for the simulated Nao from RoboCup 3D Soccer Simulation League, which is a competition where 22 humanoid simulated agents play soccer. Again, the resulting walk was fast and stable, overcoming the speed of the walk from the base team.