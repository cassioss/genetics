This work focused on using Evolutionary Algorithms (EA), more specifically Genetic Algorithms (GA), to solve different problems. Such algorithms work with populations where the smallest portion of an individual is the gene, which contains enough information for the individual to try to solve the problem. The GA developed was used for three problems: Boolean OneMax, with genes expressed by 0 or 1; Real OneMax, with genes expressed in the interval [0, 1); and a variation of the Traveling Salesman Problem, in which it's possible to go through shortcuts between the cities. In order to optimize the GA, two extra implementations were used. The first one was the elitism between generations, keeping the best individual immune to variations. The second one was using not only a static version of the GA, in which the input parameters are fixed during the execution, but also an extra module called Adaptive Genetic Algorithm (AGA), which was added to the GA's code and allows variation parameters (recombination and mutation) to change during the execution. The conclusion of this work was that, since the same GA was used to different problems (with the exception of problem-specific inputs), the treatment that would ease convergence for one problem towards the best solution(s) is not the same for other problems. In terms of optimizations, using an AGA turned out to be effective for finding good solutions. For future works, we suggest, when the focus is to solve a specific problem, to mold the GA to benefit from this choice to the fullest, and use some version of AGA in its implementation, too.
