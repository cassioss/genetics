Evolutionary Algorithms (AEs) are of great interest in solving complex problems. Based on the biological concepts of Evolution and trial-and-error resolution, they are able to obtain good answers with performance superior to many algorithms. This work promoted the implementation and analysis of a Genetic Algorithm (GA), a subset of AEs whose information is contained in genes, which compose individuals who try to solve the problems. The objective of this work was to use it in the optimized resolution of three problems: Boolean OneMax, whose genes are expressed by 0 or 1; Real OneMax, whose genes are expressed by a real variable between 0 to 1; and an adaptation of the Traveling Salesman Problem that allows shortcuts between the cities. To optimize the GA, two additions were made to the code. The first was the elitism between generations, keeping the best individual immune to variations. The second one was the implementation of an extra module called Adaptive Genetic Algorithm (AGA), added to the GA code and responsible for updating the mutation parameter according to the evolution of the population. The three problems were simulated using both the GA with the static parameters and the AGA. The following simulations brought small differences from the AGA compared to the static GA for the OneMax problemas with the same inputs, and a significantly better performance of the AGA for the Traveling Salesman Problem. These results lead us to conclude the the AGA is promising in solving problems.