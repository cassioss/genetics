\label{6_conclusoes}

\section{Conclusões}

Este trabalho teve como objetivo o desenvolvimento de um Algoritmo Genético (AG) para a resolução de diferentes problemas de modo otimizado. Foram escolhidos três problemas: OneMax Booleano (variável booleana), OneMax Real (variável real) e Caixeiro Viajante Adaptado (permitindo atalhos). De modo a melhorar a performance do AG, foram implementadas duas otimizações: o elitismo do melhor indivíduo, e o uso de um Algoritmo Genético Adaptativo (AGA) com implementação própria, focada na adaptação do parâmetro de mutação.

Para os problemas OneMax, o AGA mostrou uma performance mais baixa do que aquela mostrada para o AG com parâmetros estáticos, mas mostrou uma performance muito melhor para o problema do Caixeiro Viajante Adaptado, um problema bem mais complexo. Por conta disso, conclui-se que o AGA implementado foi capaz de otimizar o AG, e por ter uma implementação simples, o AGA tem potencial para ser melhorado ainda mais. Seu uso é fortemente considerado em outros problemas e trabalhos.

\section{Trabalhos Futuros}

O desenvolvimento feito neste trabalho resultaram na criação de um algoritmo adaptativo e de implementações próprias tanto do algoritmo genético quanto dos problemas escolhidos. A proposta deste trabalho foi audaz, e muitos conceitos foram desenvolvidos ao mesmo tempo, conceitos que precisam ser quebrados e maturados ainda mais. Sugere-se então as seguintes frentes de trabalho futuro:

\begin{itemize}

	\item Análise mais cuidadosa do problema do OneMax Real, considerando modelagens matemáticas e sua proximidade com o OneMax Booleano;

	\item Análise do problema do Caixeiro Viajante com a distribuição genética deste trabalho, comparada com implementações mais tradicionais (este trabalho distribuiu os genes deste problema de tal forma que o crossover de dois pontos ainda fosse possível);

	\item Evolução dos conceitos por trás do AGA desenvolvido aqui, buscando formalizar os conceitos por trás de seu desenvolvimento e, onde for possível, melhorá-lo (tanto para performance quanto para busca de soluções);

	\item Testar o AGA junto a outros problemas, para verificar prós, contras e limitações;

	\item Os gráficos de teste do AGA mostraram um comportamento aproximadamente periódico para a curva de $p_m$ e do desvio, mesmo com o comportamento aleatório da mutação. Pode ser interessante estudar este comportamento;

	\item O algoritmo em si foi pouco trabalhado (mesmo tendo muita coisa sendo discutida aqui), dado o tempo utilizado para este trabalho. Recomenda-se uma melhoria drástica deste algoritmo de modo a comportar outras implementações dos problemas, dos operadores de evolução e de outros algoritmos adaptativos.

\end{itemize}
