Algoritmos Evolutivos (AE) são de grande interesse na resolução de problemas complexos. Baseados nos conceitos biológicos de Evolução e na resolução por tentativa-e-erro, são capazes de obter boas respostas com performance superior a muitos algoritmos. Este trabalho promoveu a implementação e a análise de um Algoritmo Genético (AG), subgrupo dos AEs cuja informação é contida em genes, os quais compõem indivíduos que tentam resolver os problemas. O objetivo deste trabalho foi o de utilizá-lo na resolução otimizada de três problemas: OneMax Booleano, cujos genes são expressos por 0 ou 1; OneMax Real, cujos genes são expressos por uma variável real de 0 a 1; e uma adaptação do Problema do Caixeiro Viajante que permite utilizar atalhos entre as cidades. Para otimizar o AG, foram feitos dois acréscimos ao código. O primeiro deles foi o elitismo entre as gerações, mantendo o melhor indivíduo imune a variações. O segundo deles foi a implementação própria de um módulo extra chamado Algoritmo Genético Adaptativo (AGA), adicionado ao código do AG e responsável por atualizar o parâmetro de mutação de acordo com a evolução da população. Os três problemas foram simulados com uso do AG com parâmetros estáticos e com uso do AGA. As simulações decorrentes trouxeram diferenças pequenas do AGA comparado ao AG estático para os problemas OneMax com as mesmas entradas, e uma performance significativamente melhor do AGA para o problema do Caixeiro Viajante. Tais resultados nos levaram a concluir que o AGA utilizado é promissor na resolução de problemas.