Este trabalho focou no uso de Algoritmos Evolutivos (AE), mais especificamente de um Algoritmo Genético (AG), para a resolução de diferentes problemas. Tais algoritmos trabalham com populações onde a menor parte de um indivíduo dela é o gene, o qual contém informação suficiente para que o indivíduo tente resolver o problema. O AG desenvolvido foi utilizado para três problemas: OneMax Booleano, com genes expressados por 0 ou 1; OneMax Real, com genes expressados no intervalo [0, 1); e uma variação do Problema do Caixeiro Viajante que permite utilizar atalhos entre as cidades. De modo a otimizar o AG, utilizou-se duas implementações extras. A primeira delas foi a de elitismo entre as gerações, mantendo o melhor indivíduo imune a variações. A segunda foi o uso não só da versão estática do AG, na qual os parâmetros de entrada são fixos durante a execução, mas também de um módulo extra chamado Algoritmo Genético Adaptativo (AGA), o qual foi adicionado ao código do AG e permite que parâmetros de variação (recombinação e mutação) mudem durante a execução. A conclusão deste trabalho foi a de que, por se utilizar a mesma implementação do AG para diferentes problemas (com exceção das entradas específicas de cada problema), o tratamento que facilitaria a convergência de um problema para a(s) melhore(s) resposta(s) acabou não sendo o melhor para outros problemas. Em termos de otimização, o uso do AGA se mostrou eficaz no encontro de boas soluções, e deve ser dado mais atenção a ele. Para trabalhos futuros, sugere-se, quando o foco for trabalhar em um problema específico, moldar o AG para se beneficiar desta escolha ao máximo, e utilizar também alguma versão de AGA em sua implementação.
