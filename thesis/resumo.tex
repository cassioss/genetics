A extra��o de caracter�sticas musicais de sinais sonoros � uma tarefa com aplica��o em diversas �reas como compress�o de sinais, ensino, edi��o musical, etc. Todavia, extrair informa��es de um sinal sonoro gravado com instrumentos reais � uma tarefa complexa. Apesar de diversos m�todos j� terem sido desenvolvidos para extra��o de caracter�sticas como tom, intensidade e dura��o de notas, existem poucos trabalhos que unificam essas etapas com alta confiabilidade, o que se reflete na escassez de produtos comerciais capazes de interpretar musicalmente sinais sonoros de forma fiel. Este trabalho objetiva a implementa��o de um programa capaz de realizar em tempo real a transcri��o musical de sinais sonoros, mesmo que ruidosos, para nota��o musical moderna. Para tornar o problema trat�vel, o estudo foi desenvolvido visando a an�lise de sinais provenientes de instrumentos de sopro temperados e monof�nicos, embora tamb�m sejam apresentados resultados obtidos ao analisar outros tipos de instrumento. O programa desenvolvido pode ser dividido em 3 m�dulos principais, respons�veis pela an�lise de tom, intensidade e tempo. A extra��o de tom � feita utilizando uma adapta��o do algoritmo de produto espectral harm�nico (HPS), a de intensidade pelo c�lculo da raiz quadr�tica m�dia e a de tempo por algoritmos de aprendizado. Por fim um m�todo heur�stico utiliza os resultados dos 3 m�dulos para gerar um arquivo MIDI para o sinal, que pode ser facilmente convertido para nota��o musical moderna. O programa desenvolvido foi avaliado utilizando m�sicas do repert�rio cl�ssico ocidental.
