\vspace*{15.0 cm}

\textit{4.0141 There is a general rule by means of which the musician can obtain the symphony from the score, and which makes it possible to derive the symphony from the groove on the gramophone record, and, using the first rule, to derive the score again. That is what constitutes the inner similarity between these things which seem to be constructed in such entirely different ways. And that rule is the law of projection which projects the symphony into the language of musical notation. It is the rule for translating this language into the language of gramophone records.}

\begin{flushright}
Ludwig Wittgenstein. Tractatus Logico-Philosophicus.
\end{flushright}

\iffalse
\vspace*{14.0 cm}

\noindent\textit{Dear Moore,}

\indent\textit{Your letter annoyed me. When I wrote Logik I didn't consult the Regulations, and therefore I think it would only be fair if you gave me my degree without consulting them so much either! As to a Preface and Notes; I think my examiners will easily see how much I have cribbed from Bosanquet. - If I'm not worth your making an exception for me even in some STUPID details then I may as well go to Hell directly; and if I am worth it and you don't do it then - by God - you might go there.}

\indent\textit{The whole business is too beastly to go on writing about it so -}

\noindent\textit{L.W.}

\begin{flushright}
(Carta de Ludwig Wittgenstein a George Moore)
\end{flushright}
\fi